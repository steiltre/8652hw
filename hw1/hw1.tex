%        File: hw1.tex
%     Created: Tue Jan 31 07:00 AM 2017 C
% Last Change: Tue Jan 31 07:00 AM 2017 C
%

\documentclass[a4paper]{article}

\title{Math 8652 Homework 1 }
\date{2/8/17}
\author{Trevor Steil}

\usepackage{amsmath}
\usepackage{amsthm}
\usepackage{amssymb}
\usepackage{esint}
\usepackage{enumitem}
\usepackage{algorithm}
\usepackage{algorithmicx}
\usepackage{algpseudocode}
\usepackage{bbm}

\newtheorem{theorem}{Theorem}[section]
\newtheorem{corollary}{Corollary}[section]
\newtheorem{proposition}{Proposition}[section]
\newtheorem{lemma}{Lemma}[section]
\newtheorem*{claim}{Claim}
\newtheorem*{problem}{Problem}
%\newtheorem*{lemma}{Lemma}
\newtheorem{definition}{Definition}[section]

\newcommand{\R}{\mathbb{R}}
\newcommand{\N}{\mathbb{N}}
\newcommand{\C}{\mathbb{C}}
\newcommand{\Z}{\mathbb{Z}}
\newcommand{\Q}{\mathbb{Q}}
\newcommand{\E}{\mathbb{E}}
\newcommand{\supp}[1]{\mathop{\mathrm{supp}}\left(#1\right)}
\newcommand{\lip}[1]{\mathop{\mathrm{Lip}}\left(#1\right)}
\newcommand{\curl}{\mathrm{curl}}
\newcommand{\la}{\left \langle}
\newcommand{\ra}{\right \rangle}
\renewcommand{\vec}[1]{\mathbf{#1}}

\newenvironment{solution}[1][]{\emph{Solution #1}}

\algnewcommand{\Or}{\textbf{ or }}
\algnewcommand{\And}{\textbf{ or }}

\begin{document}
\maketitle

\begin{enumerate}
  \item
    \begin{problem}
      If $X$ and $Y$ are independent, show that $\E [Y|\sigma(X)] = \E[Y]$ for $Y \in L^1$.
    \end{problem}

    \begin{proof}
      We must show that $\E[Y]$ satisfies the two conditions for the uniqueness result for conditional expectations.

      First, $\E[Y]$ is a constant, so $\E[Y] \in \sigma(X)$.

      Next, we must show $\int_{A}^{} \E[Y] dP = \int_{A}^{} Y dP$ for all $A \in \sigma(X)$.

      Let $A \in \sigma(X)$. Then for any Borel measurable set $B$, $\mathbbm{1}^{-1}_A(B) \in \{ \emptyset, A, A^c, \Omega \}$. $A \in \sigma(X)$ and
      $X$ and $Y$ are independent. Therefore, $Y$ and $\mathbbm{1}_A$ are independent as well. So we have
      \begin{align*}
        \int_{A}^{} \E[Y] dP &= \E[Y] \int_{A}^{} dP \\
        &= \E[Y] P(A) \\
        &= \int_{\Omega}^{} Y \mathbbm{1}_A dP \quad \parbox{4cm}{by independence} \\
        &= \int_{A}^{} Y dP
      \end{align*}

      Therefore, by the uniqueness of conditional expectations, $\E[Y] = \E[Y | \sigma(X)]$ a.s.
    \end{proof}

  \item
    \begin{problem}
      Suppose $\E [Y | \mathcal{G}] = X$ and $\E [ Y^2 | \mathcal{G}] = X^2$ for random variables $X$ and $Y$ with $Y \in L^2$. How must $X$ and $Y$
      be related?
    \end{problem}

    \begin{solution}
      We must have that $X = Y$ a.s.

      First, we notice that $X \in \mathcal{G}$ by definition. Therefore,
      \[ \E[ XY | \mathcal{G}] = X \E[Y | \mathcal{G}] = X^2 .\]
      So for any $A \in \mathcal{G}$,
      \[ \int_{A}^{} XY dP = \int_{A}^{} X^2 dP .\]

      With this, we have
      \begin{align*}
        \E[ (X-Y)^2 ] &= \int_{\Omega}^{} (X-Y)^2 dP \\
        &= \int_{\Omega}^{} X^2 - 2XY + Y^2 dP \\
        &= \int_{\Omega}^{} 2 (X^2 - XY) dP \\
        &= 0
      \end{align*}
      because $\Omega \in \mathcal{G}$. Therefore, $X = Y$ a.s.
    \end{solution}

  \item
    \begin{problem}
      Verify the claim on page 16 of the notes that, when a regular conditional probability kernel $Q( \cdot, \cdot)$ exists for $(\Omega,
      \mathcal{F}, P)$ and $\mathcal{G}$, then, for any $Y \in \mathcal{F}$ with $\E [|Y|] < \infty$, a version of $\E[ Y | \mathcal{G}]$ is given by
      \[ \int_{\Omega}^{} Y(\omega') Q(\omega, d \omega') .\]
    \end{problem}

    \begin{proof}
      We must show the two properties that define versions of conditional expectations.

      First, assume $Y = \sum_{i=1}^n c_i \mathbbm{1}_{A_i}$ is a simple function. Then for any $A \in \mathcal{G}$
      \begin{align*}
        \int_{A}^{} \int_{\Omega}^{} Y(\omega') Q(\omega, d\omega') dP &= \int_{A}^{} \sum_{i=1}^n c_i \int_{\Omega}^{} \mathbbm{1}_{A_i} (\omega')
        Q(\omega, d\omega') dP \\
        &= \sum_{i=1}^n c_i \int_{A}^{} Q(\omega, A_i) dP \\
        &= \sum_{i=1}^n c_i P(A \cup A_i) \\
        &= \sum_{i=1}^n c_i \int_{A}^{} \mathbbm{1}_{A_i} dP \\
        &= \int_{A}^{} Y dP
      \end{align*}

      Also,
      \begin{align*}
        \int_{\Omega}^{} Y(\omega') Q(\omega, d\omega') &= \sum_{i=1}^n c_i \int_{\Omega}^{} \mathbbm{1}_{A_i}(\omega') Q(\omega, d\omega') \\
        &= \sum_{i=1}^n c_i Q(\omega, A_i)
      \end{align*}
      is measurable because it is a finite sum of measurable functions (of $\omega$).

      For general $Y$, we split $Y$ into its positive and negative components. For simplicity, we will assume $Y \geq 0$ and only show the argument
      for the positive piece, but the same argument holds for the negative component.

      Let $Y_n$ be an increasing sequence of simple functions such that $Y_n \to Y$ a.s. Because $Y \geq 0$ and $Q(\omega, A) \in [0,1]$, we have an
      increasing sequence of functions and can apply the monotone convergence theorem to get
      $\int_{\Omega}^{} Y(\omega') Q(\omega, d\omega')$ is measurable, and for any $A \in \mathcal{G}$,
      \begin{align*}
        \int_{A}^{} \int_{\Omega}^{} Y(\omega') Q(\omega, d\omega') dP &= \lim_{n \to \infty} \int_{A}^{} \int_{\Omega}^{} Y_n (\omega') Q(\omega,
        d\omega') \\
        &= \lim_{n \to \infty} \int_{A}^{} Y_n dP \\
        &= \int_{A}^{} Y dP
      \end{align*}

      Thus, $\int_{\Omega}^{} Y(\omega') Q(\omega, d\omega')$ satisfies the necessary conditions and is a version of $\E[Y|\mathcal{G}]$.
    \end{proof}

  \item
    \begin{problem}
      Given $(\Omega, \mathcal{F}, P)$, suppose $\mathcal{G}$ and $\mathcal{H}$ are sub $\sigma$-fields of $\mathcal{F}$. Is it necessarily true that,
      for $X \in \mathcal{F}$ with $X \in L^1$,
      \[ \E [ \E[X|\mathcal{G}] | \mathcal{H} ] = \E [ \E[X|\mathcal{H}] | \mathcal{G} ] ? \]
    \end{problem}

    \begin{solution}

      This is not true. Let $\Omega = \{a, b, c\}$, $\mathcal{F} = 2^\Omega$, and $P$ be the counting measure. Define the random variable $X$ by
      \[ X(t) =
        \begin{cases}
          1 &\text{if } t = a \\
          2 &\text{if } t = b \\
          3 &\text{if } t = c \\
        \end{cases}
      \]

      Let $\mathcal{G} = \{ \emptyset, \{a\}, \{b,c\}, \Omega \}$ and $\mathcal{H} = \{ \emptyset, \{b\}, \{a,c\}, \Omega \}$.

      First, we will calculate $\E[X|\mathcal{G}]$. We must have
      $\int_{\{a\}}^{} \E[ X|\mathcal{G}] dP = \int_{\{a\}}^{} X dP = 1$. Therefore, $\E[X|\mathcal{G}](a) = 1$.
      We must also have
      $\int_{\{b,c\}}^{} \E[ X|\mathcal{G} ] dP = \int_{\{b,c\}}^{} X dP = 5$. Therefore, $\E[X|\mathcal{G}](b) + \E[X|\mathcal{G}](c) = 5$. In order
      to have $\E[X|\mathcal{G}] \in \mathcal{G}$, we need $\E[X|\mathcal{G}](b) = \E[X|\mathcal{G}](c) = 2.5$.

      To summarize,
      \[ \E[X|\mathcal{G}](t) =
      \begin{cases}
        1 &\text{if } t = a \\
        2.5 &\text{if } t \in \{b,c\}
      \end{cases}
      \]

      Using the same reasoning, we get
      \[ \E[X|\mathcal{H}](t) = 2 \]
      for all $t$.

      By repeating these calculations on the new random variables we have created, we get
      \[ \E[ \E[X | \mathcal{G}] | \mathcal{H}](t) =
      \begin{cases}
        1.75 &\text{if } t \in \{a,c\} \\
        2.5 &\text{if } t = b
      \end{cases}
      \]
      and
      \[ \E[ \E[X | \mathcal{H} | \mathcal{G}](t) = 2 \quad \text{for all } t. \]
      Therefore, $\E[ \E[X|\mathcal{G}] | \mathcal{H}] \neq \E[ \E[X| \mathcal{H}] | \mathcal{G}]$.

    \end{solution}
\end{enumerate}
\end{document}


