%        File: hw4.tex
%     Created: Wed Apr 19 03:00 PM 2017 C
% Last Change: Wed Apr 19 03:00 PM 2017 C
%

\documentclass[11pt]{article}

\title{Math 8652 Homework 4 }
\date{4/26/17}
\author{Trevor Steil}

\usepackage{amsmath}
\usepackage{amsthm}
\usepackage{amssymb}
\usepackage[margin=1.0in]{geometry}
\usepackage{esint}
\usepackage{enumitem}
\usepackage{algorithm}
\usepackage{algorithmicx}
\usepackage{algpseudocode}
\usepackage{bbm}
\usepackage{xcolor}

\newtheorem{theorem}{Theorem}[section]
\newtheorem{corollary}{Corollary}[section]
\newtheorem{proposition}{Proposition}[section]
\newtheorem{lemma}{Lemma}[section]
\newtheorem*{claim}{Claim}
%\newtheorem*{problem}{Problem}
%\newtheorem*{lemma}{Lemma}
\newtheorem{definition}{Definition}[section]

\newcommand{\R}{\mathbb{R}}
\newcommand{\N}{\mathbb{N}}
\newcommand{\C}{\mathbb{C}}
\newcommand{\Z}{\mathbb{Z}}
\newcommand{\Q}{\mathbb{Q}}
\newcommand{\E}{\mathbb{E}}
\newcommand{\supp}[1]{\mathop{\mathrm{supp}}\left(#1\right)}
\newcommand{\lip}[1]{\mathop{\mathrm{Lip}}\left(#1\right)}
\newcommand{\curl}{\mathrm{curl}}
\newcommand{\la}{\left \langle}
\newcommand{\ra}{\right \rangle}
\renewcommand{\vec}[1]{\mathbf{#1}}
\renewcommand{\div}{\mathrm{div}}

\newenvironment{problem}{\textbf{Problem.}}

\newenvironment{solution}[1][]{\emph{Solution #1}}

\algnewcommand{\Or}{\textbf{ or }}
\algnewcommand{\And}{\textbf{ or }}

% Indent paragraphs and remove spacing between paragraphs in itemize and enumerate environments
\setlist{ listparindent=\parindent, parsep=0pt,}


\begin{document}
\maketitle

\begin{enumerate}
  \item
    \begin{problem}
      Suppose $S_n = X_1 + \dots + X_n$, where $X_k$ are symmetric IID random variables with $P(X_k=1) = \frac{1}{3}$ and $P(X_k \leq x) =
      \frac{x+1}{6}$ for $x \in [4,5]$. Estimate $P(S_{200} \in [0,2])$. Does the density of $S_{200}$ exist at $x = 0$? If so, compute it.
    \end{problem}

    \begin{solution}

      Because $X$ has mean zero and is symmetric, we can calculate
      \begin{align*}
        \sigma^2 &= \frac{2}{3} + 2 \int_{4}^{5} \frac{1}{6} x^2 dx \\
        &= \frac{2}{3} + \frac{61}{9} \\
        &= \frac{67}{9}
      \end{align*}

      $X_k$ takes values between -5 and 5. Therefore, $0 < \sigma^2 < \infty$. Applying the local central limit theorem with $x_n = 0$ for all $n$ and
      $a=0, b=2$, we have
      \[ \sqrt{n} P (S_n \in [a,b]) \to (b-a) .\]
      Therefore, we estimate
      \[ P(S_{200} \in [0,2]) \approx 20 \sqrt{2} .\]

    \end{solution}

  \item
    \begin{problem}
      Suppose $X_1, X_2, \dots$ are IID Bernoulli random variables with unknown mean $p$. There are two competing hypotheses, that $p= .25$ and that
      $p = .253$, and you are given $\sum_{k=1}^{10^6} X_k$. What cutoff for this sum should you use for selecting one value of $p$ or the other?
      Using this cutoff, what is the approximate probability of you being wrong?
    \end{problem}

    \begin{solution}

      Let $H_1$ be the hypothesis $p = .25$. Let $H_2$ be the hypothesis $p = .253$. We want a $\lambda$ that allows us to choose one of these
      hypotheses based on whether $S_n = \sum_{k=1}^n X_k > \lambda n$ or not.

      Using large deviations, the probability of being incorrect with
      the conclusion that $p = .25$ is
      \[ P( S_n > \lambda n | H_1 ) \approx (\varphi_{0.25}(\lambda))^n \]
      because $0.25$ corresponds to the hypothesis that $p$ is smaller. Similarly,
      \[ P( S_n < \lambda n | H_2 ) \approx (\varphi_{0.253}(\lambda))^n .\]

      We want these probabilities of error in either direction to be approximately equal, so we set
      \begin{align*}
        \varphi_{.25} &= \varphi_{.253} \\
        \left( \frac{.25}{\lambda} \right)^\lambda \left( \frac{.75}{1-\lambda} \right)^{1-\lambda} &= \left( \frac{.253}{\lambda} \right)^\lambda
        \left( \frac{.747}{1-\lambda} \right)^{1-\lambda}
      \end{align*}

      Solving for $\lambda$, we have
      \[ \lambda = \frac{\log \left(\frac{.75}{.747} \right)}{ \log \left( \frac{.253}{.25} \right) + \log \left( \frac{.75}{.747} \right) } \approx
      .2515 .\]

      Because $.253 = .25 + .003$, we can use an approximation from class to say
      \[ \log \varphi(\lambda) = \frac{-2(.003)^2}{3} + O(t^3) .\]
      Therefore, with this $\lambda$, our probability of being incorrect is approximately given by $e^{\frac{-2(.003)^2}{3} n}$. With the value of $n$
      given, this is $e^{\frac{-2(.003)^2}{3} \cdot 10^6} \approx .0025$.

    \end{solution}

  \item
    \begin{problem}
      In each of the following two cases, is there a Markov chain such that the return times to a fixed state $i$ are given by the following sequence?
      \begin{enumerate}
        \item 10, 16, 20, 22, 26, 29, 30, 31, 32, and so on, in increments of 1.
        \item 12, 18, 21, 24, 28, 32, 36, 39, 42, and so on, in increments of 3.
      \end{enumerate}
      Draw a flow chart when true.
    \end{problem}

    \begin{solution}

      \begin{enumerate}
        \item It is possible to have these return times for a Markov chain. See the attachment at the end to see a possible flow chart.

        \item This is not a possible set of return times. The greatest common divisor of these times is 1. By a result from class, we know a Markov
          chain with a state $i$ that has period 1 will satisfy $p_{ii}^n > 0$ for large enough $n$. This sequence has increments of 3, so it cannot
          be the return times for a state in a Markov chain.
      \end{enumerate}

    \end{solution}

  \item
    \begin{problem}
      Let $S_n$ be a simple symmetric random walk on $\Z$ with $S_0 = 3$. What is the expected amount of time it takes $S_n$ to exit the interval
      $(0,7)$? (Hint: consider martingales and functions of the form $ax^2 + bx - n$.)
    \end{problem}

    \begin{solution}

      Define the random variable
      \[ Y_n = S_n( S_n - 7) - n = S_n^2 - 7S_n - n .\]

      Let $\mathcal{F}_n$ be the natural history. Then
      \begin{align*}
        \E[ Y_n | \mathcal{F}_{n-1} ] &= \frac{1}{2} (S_{n-1} - 1)^2 + \frac{1}{2} (S_{n-1} + 1)^2 - \frac{7}{2} (S_{n-1} - 1) - \frac{7}{2} (S_{n-1}
        + 1) - n \\
        &= S_{n-1}^2 + 1 - 7 S_{n-1} - n \\
        &= S_{n-1}^2 - 7 S_{n-1} - (n-1) \\
        &= Y_{n-1}.
      \end{align*}

      Therefore, $\{Y_n, \mathcal{F}_n \}$ is a martingale. We are interested in when $S_n \in \{0, 7\}$, which corresponds to $Y_n = -n$. With this
      motivation, we define the stopping time
      \[ T = \inf \{ n : Y_n = -n \} .\]
      We notice that by definition of $T$, $Y_T = -T$. As was done in class, we can apply the Optional Sampling Theorem to symmetric random walks by
      applying the cutoff $T \wedge n$. This gives
      \begin{align*}
        \E[T] &= - \E[Y_T] \\
        &= - \E[Y_0] \\
        &= - S_0 (S_0 - 7) \\
        &= 12
      \end{align*}

    \end{solution}
\end{enumerate}
\end{document}


