%        File: hw3.tex
%     Created: Tue Mar 21 10:00 AM 2017 C
% Last Change: Tue Mar 21 10:00 AM 2017 C
%

\documentclass[11pt]{article}

\title{Math 8652 Homework 3 }
\date{3/27/17}
\author{Trevor Steil}

\usepackage{amsmath}
\usepackage{amsthm}
\usepackage{amssymb}
\usepackage{esint}
\usepackage{enumitem}
\usepackage{algorithm}
\usepackage{algorithmicx}
\usepackage{algpseudocode}
\usepackage{bbm}
\usepackage{xcolor}

\newtheorem{theorem}{Theorem}[section]
\newtheorem{corollary}{Corollary}[section]
\newtheorem{proposition}{Proposition}[section]
\newtheorem{lemma}{Lemma}[section]
\newtheorem*{claim}{Claim}
%\newtheorem*{problem}{Problem}
%\newtheorem*{lemma}{Lemma}
\newtheorem{definition}{Definition}[section]

\newcommand{\R}{\mathbb{R}}
\newcommand{\N}{\mathbb{N}}
\newcommand{\C}{\mathbb{C}}
\newcommand{\Z}{\mathbb{Z}}
\newcommand{\Q}{\mathbb{Q}}
\newcommand{\E}{\mathbb{E}}
\newcommand{\supp}[1]{\mathop{\mathrm{supp}}\left(#1\right)}
\newcommand{\lip}[1]{\mathop{\mathrm{Lip}}\left(#1\right)}
\newcommand{\curl}{\mathrm{curl}}
\newcommand{\la}{\left \langle}
\newcommand{\ra}{\right \rangle}
\renewcommand{\vec}[1]{\mathbf{#1}}
\renewcommand{\div}{\mathrm{div}}

\newenvironment{problem}{\textbf{Problem.}}

\newenvironment{solution}[1][]{\emph{Solution #1}}

\algnewcommand{\Or}{\textbf{ or }}
\algnewcommand{\And}{\textbf{ or }}

\begin{document}
\maketitle

\begin{enumerate}
  \item
    \begin{problem}
      Let $X_1, X_2, \dots$ be independent random variables, and set $S_n = \sum_{k=1}^n X_k$. Are the $\sigma$-fields generated by $\limsup_n S_n$
      and $\liminf_n S_n$ necessarily trivial? Why or why not?
    \end{problem}

    \begin{solution}

      The $\sigma$-fields are not necessarily trivial. Let $X_1 = 1$ with probability $\frac{1}{2}$ and $X_1 = -1$ with probability $\frac{1}{2}$. Let $X_i$ be identically zero for $i > 1$. In this
      case, $S_n = X_1$ for all $n$. So the $\sigma$-fields generated by $\limsup_n S_n$ and $\liminf_n S_n$ are the same as the $\sigma$-field
      generated by $X_1$, which is nontrivial.

    \end{solution}

  \item
    \begin{problem}
      Consider an urn that initially contains 10 black balls and 5 red balls. Each time a ball is drawn from the urn, it is replaced by itself
      together with 3 additional balls of the same color. (The number of balls in the urn is therefore increasing.) Let $Z_n = \frac{R_n}{B_n + R_n}$
      be the proportion of red balls at time $n$. Does $Z_n$ have a limit as $n \to \infty$? If so, what is its expectation?
    \end{problem}

    \begin{solution}

      Let $\mathcal{F}_n$ be the natural history. Then $\{ Z_n, \mathcal{F}_n \}$ is a martingale.

      At time $n$, the probability of drawing a red ball is $\frac{R_n}{B_n + R_n}$. If this happens,
      \[ Z_{n+1} = \frac{R_n + 3}{B_n + R_n + 3} .\]
      Likewise, the probability of drawing a black ball is $\frac{B_n}{B_n + R_n}$, and the result would be
      \[ Z_{n+1} = \frac{R_n}{B_n + R_n + 3} .\]

      Therefore,
      \begin{align*}
        \E[Z_{n+1} | \mathcal{F}_n] &= \frac{R_n + 3}{B_n + R_n + 3} \frac{R_n}{B_n + R_n} + \frac{R_n}{B_n + R_n + 3} \frac{B_n}{B_n + R_n} \\
        &= \frac{R_n (B_n + R_n + 3)}{(B_n + R_n) (B_n + R_n + 3)} \\
        &= \frac{R_n}{B_n + R_n} \\
        &= Z_n
      \end{align*}

      Notice that $Z_n$ is bounded by 1 for all $n$ and is therefore in $L^1$. Because $Z_n$ is an $L^1$ martingale, it converges almost surely. Also, for any $n$, $\E[ Z_{n+1} | \mathcal{F}_0 ] = Z_0$, so
      \begin{align*}
        \E[Z_n] &= \E[ \E[ Z_n | \mathcal{F}_0 ]] \\
        &= \E[ Z_0 ] \\
        &= \frac{1}{3}
      \end{align*}

      Therefore, $\lim_{n \to \infty} \E[ Z_n ] = \frac{1}{3}$.

    \end{solution}

  \item
    \begin{problem}
      Let $\{ X_n, \mathcal{F}_n \}$ be a supermartingale. Show that $X_n$ uniformly integrable implies $X_n$ is closed. Is the converse true? (You
      may use results proved in class.)
    \end{problem}

    \begin{solution}

      Because $X_n$ is uniformly integrable, we know $X_n \to X$ a.s. and in $L^1$. By an inductive argument, we know $\E[ X_{n+m} | \mathcal{F}_n]
      \leq X_n$ for any $m \geq 0$. For any $n$ and any $A \in \mathcal{F}_n$, we then have
      \begin{align*}
        X_n \geq \int_{A}^{} X_{n+m} dP
      \end{align*}

      By the $L^1$ convergence, we have
      \begin{align*}
        X_n &\geq \lim_{m \to \infty} \int_{A}^{} X_{n+m} dP \\
        &= \int_{A}^{} X dP
      \end{align*}

      Therefore,
      \[ X_n \geq \E[X | \mathcal{F}_n], \]
      and $\{X_n, \mathcal{F}_n\}$ is a closed supermartingale.

      The converse is not always true. Consider ``double or nothing'', that is, let $\xi_i$ be independent random variables with $P(\xi_i = 0) =
      P(\xi_i = 2) = \frac{1}{2}$. Define
      \[ X_n = \prod_{i=1}^n \xi_i \]
      and let $\mathcal{F}_n$ be the natural history.

      $\{ X_n, \mathcal{F}_n \}$ is a martingale and therefore is also a supermartingale. $X_n \geq 0$, so it is closed by 0 as a supermartingale.

      For any $n$, $\E[X_n] = 1$. If we let $A_n = \{X_n > 0\}$, we have $P(A_n) = 2^{-n}$, and $X = 2^n$ on $A_n$. So for any $\delta$, let $n$ be
      such that $\delta > 2^{-n}$. Then we have $P(A_n) < \delta$, but $\int_{A_n}^{} |X_n| dP > \frac{1}{2}$. Thus, $\{ X_n \}$ is not uniformly
      integrable.

    \end{solution}

  \item
    \begin{problem}
      Let $\{ X_n, \mathcal{F}_n \}$ be a positive reversed submartingale. Then, does $\lim_{n \to -\infty} X_n$ exist (as a finite limit)? Same
      question, but where $\{ X_n, \mathcal{F}_n \}$ is instead a supermartingale.
    \end{problem}

    \begin{solution}

      By the Doob decomposition, we can write $X_n = M_n + A_n$ where $M_n$ is a reversed martingale and $A_n$ decreases as $n \to -\infty$ and is
      predictable. (We only proved this for supermartingales, but the same proof holds for reversed submartingales.)

      Because $M_n$ is a reversed martingale, $M_n \to M$ a.s. by the reversed martingale convergence theorem.

      If $A_n < -M_n$ for any $n$ (at any point $\omega \in \Omega$), then $X_n = M_n + A_n < 0$, which contradicts the positivity of $X_n$.
      Therefore, $A_n \geq -M_n$. Since $M_n$ converges to a finite limit a.s., $A_n$ must be bounded from below a.s. Therefore, $A_n$ is a decreasing sequence
      (as $n \to -\infty$) which is bounded from below a.s. Thus, $A_n$ converges a.s.

      Because $M_n$ and $A_n$ converge to finite limits a.s., $X_n$ does as well.

      This result does not hold for positive reversed supermartingales. Let $X_n = -n$. Then
      \[ \E[ X_{n+1} | \mathcal{F}_n ] = -(n+1) \leq n = X_n .\]
      For negative values of $n$, $X_n$ is clearly positive. However, $X_n \to \infty$ as $n \to -\infty$.
    \end{solution}
\end{enumerate}
\end{document}


